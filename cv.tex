%%%%%%%%%%%%%%%%%%%%%%%%%%%%%%%%%%%%%%%%%
% Wilson Resume/CV
% XeLaTeX Template
% Version 1.0 (22/1/2015)
%
% This template has been downloaded from:
% http://www.LaTeXTemplates.com
%
% Original author:
% Howard Wilson (https://github.com/watsonbox/cv_template_2004) with
% extensive modifications by Vel (vel@latextemplates.com)
%
% License:
% CC BY-NC-SA 3.0 (http://creativecommons.org/licenses/by-nc-sa/3.0/)
%
%%%%%%%%%%%%%%%%%%%%%%%%%%%%%%%%%%%%%%%%%

%----------------------------------------------------------------------------------------
%	PACKAGES AND OTHER DOCUMENT CONFIGURATIONS
%----------------------------------------------------------------------------------------

\documentclass[10pt]{article} % Default font size

%%%%%%%%%%%%%%%%%%%%%%%%%%%%%%%%%%%%%%%%%
% Wilson Resume/CV
% Structure Specification File
% Version 1.0 (22/1/2015)
%
% This file has been downloaded from:
% http://www.LaTeXTemplates.com
%
% License:
% CC BY-NC-SA 3.0 (http://creativecommons.org/licenses/by-nc-sa/3.0/)
%
%%%%%%%%%%%%%%%%%%%%%%%%%%%%%%%%%%%%%%%%%

%----------------------------------------------------------------------------------------
%	PACKAGES AND OTHER DOCUMENT CONFIGURATIONS
%----------------------------------------------------------------------------------------

\usepackage[a4paper, hmargin=25mm, vmargin=30mm, top=20mm]{geometry} % Use A4 paper and set margins

\usepackage{fancyhdr} % Customize the header and footer

\usepackage{lastpage} % Required for calculating the number of pages in the document

\usepackage{hyperref} % Colors for links, text and headings

\setcounter{secnumdepth}{0} % Suppress section numbering

%\usepackage[proportional,scaled=1.064]{erewhon} % Use the Erewhon font
%\usepackage[erewhon,vvarbb,bigdelims]{newtxmath} % Use the Erewhon font
\usepackage[utf8]{inputenc} % Required for inputting international characters
\usepackage[T1]{fontenc} % Output font encoding for international characters

\usepackage{fontspec} % Required for specification of custom fonts
\setmainfont[Path = ./fonts/,
Extension = .otf,
BoldFont = Erewhon-Bold,
ItalicFont = Erewhon-Italic,
BoldItalicFont = Erewhon-BoldItalic,
SmallCapsFeatures = {Letters = SmallCaps}
]{Erewhon-Regular}

\usepackage{color} % Required for custom colors
\definecolor{slateblue}{rgb}{0.17,0.22,0.34}

\usepackage{sectsty} % Allows customization of titles
\sectionfont{\color{slateblue}} % Color section titles

\fancypagestyle{plain}{\fancyhf{}\cfoot{\thepage\ of \pageref{LastPage}}} % Define a custom page style
\pagestyle{plain} % Use the custom page style through the document
\renewcommand{\headrulewidth}{0pt} % Disable the default header rule
\renewcommand{\footrulewidth}{0pt} % Disable the default footer rule

\setlength\parindent{0pt} % Stop paragraph indentation

% Non-indenting itemize
\newenvironment{itemize-noindent}
{\setlength{\leftmargini}{0em}\begin{itemize}}
{\end{itemize}}

% Text width for tabbing environments
\newlength{\smallertextwidth}
\setlength{\smallertextwidth}{\textwidth}
\addtolength{\smallertextwidth}{-2cm}

\newcommand{\sqbullet}{~\vrule height 1ex width .8ex depth -.2ex} % Custom square bullet point definition

%----------------------------------------------------------------------------------------
%	MAIN HEADER COMMAND
%----------------------------------------------------------------------------------------

\renewcommand{\title}[1]{
{\huge{\color{slateblue}\textbf{#1}}}\\ % Header section name and color
\rule{\textwidth}{0.5mm}\\ % Rule under the header
}

%----------------------------------------------------------------------------------------
%	JOB COMMAND
%----------------------------------------------------------------------------------------

\newcommand{\job}[6]{
\begin{tabbing}
\hspace{2cm} \= \kill
\textbf{#1} \> \href{#4}{#3} \\
\textbf{#2} \>\+ \textit{#5} \\
\begin{minipage}{\smallertextwidth}
\vspace{2mm}
#6
\end{minipage}
\end{tabbing}
\vspace{2mm}
}

%----------------------------------------------------------------------------------------
%	SKILL GROUP COMMAND
%----------------------------------------------------------------------------------------

\newcommand{\skillgroup}[2]{
\begin{tabbing}
\hspace{5mm} \= \kill
\sqbullet \>\+ \textbf{#1} \\
\begin{minipage}{\smallertextwidth}
\vspace{2mm}
#2
\end{minipage}
\end{tabbing}
}

%----------------------------------------------------------------------------------------
%	INTERESTS GROUP COMMAND
%-----------------------------------------------------------------------------------------

\newcommand{\interestsgroup}[1]{
\begin{tabbing}
\hspace{5mm} \= \kill
#1
\end{tabbing}
\vspace{-10mm}
}

\newcommand{\interest}[1]{\sqbullet \> \textbf{#1}\\[3pt]} % Define a custom command for individual interests

%----------------------------------------------------------------------------------------
%	TABBED BLOCK COMMAND
%----------------------------------------------------------------------------------------

\newcommand{\tabbedblock}[1]{
\begin{tabbing}
\hspace{2cm} \= \hspace{4cm} \= \kill
#1
\end{tabbing}
} % Include the file specifying document layout

%----------------------------------------------------------------------------------------

\begin{document}

%----------------------------------------------------------------------------------------
%	NAME AND CONTACT INFORMATION
%----------------------------------------------------------------------------------------

\title{CV -- Alexandre Mercier-Aubin} % Print the main header

%------------------------------------------------

\parbox{0.5\textwidth}{ % First block
\begin{tabbing} % Enables tabbing
\hspace{3cm} \= \hspace{4cm} \= \kill % Spacing within the block
{\bf Address} \> 73 Barrette,\\ % Address line 1
\> Levis, Quebec, Canada \\ % Address line 2
{\bf Date of Birth} \> December 16, 1996 \\ % Date of birth 
{\bf Nationality} \> Canadian % Nationality
\end{tabbing}}
\hfill % Horizontal space between the two blocks
\parbox{0.5\textwidth}{ % Second block
\begin{tabbing} % Enables tabbing
\hspace{3cm} \= \hspace{4cm} \= \kill % Spacing within the block
{\bf Mobile Phone} \> +1 (418) 572 0698 \\ % Mobile phone
{\bf Email} \> \href{mailto:alexandre.mercier-aubin.1@ulaval.ca}{alexandre.mercier-aubin.1@ulaval.ca} \\ % Email address
{\bf Website} \> \href{alexandremercieraubin.github.io}{alexandremercieraubin.com} \\
\end{tabbing}}

%----------------------------------------------------------------------------------------
%	PERSONAL PROFILE
%----------------------------------------------------------------------------------------

\section{Personal Profile}

Graduate student at McGill University in computer science. My fields of study are constraint programming, computer graphics and algorithms. I seek knowledge as well as new challenges to push forward my limits.

%----------------------------------------------------------------------------------------
%	EDUCATION SECTION
%----------------------------------------------------------------------------------------

\section{Education}

\tabbedblock{
\bf{2020-2023} \> PhD in Computer Science - \href{https://www.ulaval.ca/en/}{McGill University} \\[5pt]
\>\+
\textit{In progress} \> \\
}

%------------------------------------------------

\tabbedblock{
\bf{2019-2020} \> M. Sc. in Computer Science - \href{https://www.ulaval.ca/en/}{Université Laval} \\[5pt]
\>\+
\textit{Research in constraint programming with Prof. Claude-Guy Quimper} \> \\
}

\tabbedblock{
\bf{2017-2019} \> B. Sc. in Computer Science - \href{https://www.ulaval.ca/en/}{Université Laval}\\[5pt]
\>\+
\textit{honor profile} \> \\
}

\tabbedblock{
\bf{2014-2017} \> DEC-BAC in Computer Science - \href{http://cll.qc.ca/}{Cégep Lévis-Lauzon}\\[5pt]
\>\+
}

%----------------------------------------------------------------------------------------
%	EMPLOYMENT HISTORY SECTION
%----------------------------------------------------------------------------------------

\section{Employment History}
\job
{}{Present}
{McGill}
{mcgill.ca}
{teaching assistant and AGSEM delegate}
{Attend office hours and defend the rights of TAs}

\job
{}{2020}
{Université Laval}
{https://www.ulaval.ca/en/}
{teaching assistant}
{Teach the practical aspect of the advanced programming in C++ course. Corrector for the computer graphics course. Support students at the help center for computer science students (CARÉ) with questions related to 15 different courses: from beginner programming courses to advanced algorithm design and operating systems.}

\job
{2018/05,}{2018/09}
{Activision, Beenox}
{https://www.activision.com/}
{game engine developper intern}
{Design and program the game engine for Call of Duty: Black Ops 4.\\
 Technologies: DirectX, C++, LUA 
 Contributions: Shaders, LOD formulas, Bugfixes, HUD formulas, cross-platform compatibilities, etc.}

%------------------------------------------------

\job
{2017/01 -}{2017/08}
{Centre de Robotic et Vision Industrielle}
{http://www.crvi.ca/}
{Intern/Programmer}
{Machine learning applied to Computer Vision.}

\job
{2016/05 -}{2016/09}
{Valero, Levis}
{https://www.valero.com}
{computer science intern}
{Translate programs from Visual Basic to C\#.}

\job
{2015/05 -}{2015/09}
{Consortium Cooperatif de Services Fédératifs}
{https://leconsortium.coop/en/}
{IT Technician}
{IT support, creating and managing a database, creating a web site, etc.}

%----------------------------------------------------------------------------------------
%	IT/COMPUTING SKILLS SECTION
%----------------------------------------------------------------------------------------

\section{Prizes and scholarship}

\skillgroup{Bourse de doctorat Hydro-Québec en science}
{
 15 000\$ \textit{per year, up to 3 years} \\
}

\skillgroup{School of Computer Science PhD funding, McGill University}
{
 21 000\$ \textit{per year, up to 3 years} \\
 3 000\$ top up from Mechanical Engineering.\\
}

\skillgroup{MITACS accelerate, CRISI}
{
\textit{} 39 000\$\\
}

\skillgroup{Undergraduate Research Fellowship 2019-2020, Université Laval}
{
\textit{I declined} 6500\$\\
}
%------------------------------------------------

\skillgroup{Association for Constraint Programming, CP2019}
{
\textit{} 450\$\\
}

\section{Publications}
\subsection{Peer-reviewed conference papers}
\subsubsection{published}
\begin{itemize}

\item
Alexandre Mercier-Aubin, Ludwig Dumetz, Jonathan Gaudreault, and Claude-Guy Quimper. The Confidence Constraint : A Step Towards Stochastic CP Solvers. In Proceedings of the 26th International Conference on Principles and Practice of Constraint Programming (CP 2020), pages 759-773, 2020.

\item
Alexandre Mercier-Aubin, Jonathan Gaudreault, and Claude-Guy Quimper. Leveraging Constraint Scheduling: A Case Study to the Textile Industry. In Proceedings of the 17th International Conference on the Integration of Constraint Programming, Artificial Intelligence, and Operations Research (CPAIOR 2020), pages 334-346, 2020.

\end{itemize}

\subsection{Thesis}
\begin{itemize}
\item
Alexandre Mercier-Aubin. Ordonnancement de tâches sous contraintes sur des métiers à tisser. M. Sc. Thesis, Université Laval, 2020.
\end{itemize}

\subsection{Peer-reviewed workshops}
\begin{itemize}

\item
Alexandre Mercier-Aubin, Jonathan Gaudreault and Claude-Guy Quimper. Multi-Resource Scheduling with Setup Times: An Application Case to the Textile Industry. 25th International Conference on Principles and Practice of Constraint Programming (CP 2019): Doctoral program.
\url{https://cp2019.a4cp.org/accepted_dp/09-alexandre_mercier_aubin.pdf} (accessed January 29, 2020).

\end{itemize}


%----------------------------------------------------------------------------------------
%	INTERESTS SECTION
%----------------------------------------------------------------------------------------

\section{Interests}

\interestsgroup{
\interest{Teaching}
\interest{Algorithms}
\interest{Games \& game development}
\interest{Constraint programming}
\interest{Hiking}
\interest{Music}
}

\section{Leadership}
\begin{itemize}
\item \textbf{Leader in various undergraduate team projects}: \href{https://github.com/AlexandreMercierAubin/Gaudrophone}{Gaudrophone}, \href{https://github.com/AlexandreMercierAubin/BigData7027}{BigData7027}, \href{https://github.com/AlexandreMercierAubin/projet2018-eq4}{projet2018-eq4} and \href{https://github.com/AlexandreMercierAubin/ComputerGraphics}{ComputerGraphics}, etc. \\ 
\item \textbf{Team leader in the Valleyfield game creator competition}
\end{itemize}

\section{Some interesting projects}

\skillgroup{Game engines}
{
\textbf{One made during college}: https://youtu.be/8OPpt3-iIbY\\
\textbf{The source code of a more complex engine}: https://github.com/AlexandreMercierAubin/ComputerGraphics\\

}

\skillgroup{Video games I worked on}
{
\textbf{Call of Duty: Black Ops 4}: https://www.callofduty.com/ca/en/blackops4

\textbf{Proto-Spyder Assault, 48h Valleyfield game dev contest}: https://youtu.be/qJjy8b0kuSY

\textbf{SansFin, french horror game, collegial project}: https://youtu.be/s6vr07Nt1IY
}

\section{Extracurricular}
\textbf{AGSEM Delegate}: Act as a delegate for the department of Computer Science at McGill.

\textbf{V.P. Social at the AGIL}: Organize events for the association of graduate student in computer science.

\textbf{Volunteer at the ASETIN}: Volunteer work at the student association of computer science. I've been a bartender, receptionist and helped prepare the initiations.

\textbf{Volunteer at Lévis}: Receptionist at the Water Festival of Lévis.

\textbf{Student volunteer at SCP 2020}: I support the week-­long conference by ensuring the smooth functioning of operations in technical paper sessions and main exhibitions.
%----------------------------------------------------------------------------------------
%	REFEREE SECTION
%----------------------------------------------------------------------------------------

\section{Referees}

\parbox{0.5\textwidth}{ 
\begin{tabbing}
\hspace{2.75cm} \= \hspace{4cm} \= \kill 
{\bf Name} \> Claude-Guy Quimper \\ 
{\bf Company} \> Université Laval \\ 
{\bf Position} \> Associate Professor \\  
{\bf Contact} \> \href{mailto:Claude-Guy.Quimper@ift.ulaval.ca}{Claude-Guy.Quimper@ift.ulaval.ca} 
\end{tabbing}}
\hfill 
\parbox{0.5\textwidth}{
\begin{tabbing}
\hspace{2.75cm} \= \hspace{4cm} \= \kill
{\bf Name} \> Jonathan Gaudreault\\ 
{\bf Company} \> Université Laval \\ 
{\bf Position} \> Full Professor \\ 
{\bf Contact} \> \href{mailto:jonathan.gaudreault@ift.ulaval.ca}{jonathan.gaudreault@ift.ulaval.ca}
\end{tabbing}}

%----------------------------------------------------------------------------------------

\end{document}