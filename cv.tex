%----------------------------------------------------------------------------------------
%	PACKAGES AND OTHER DOCUMENT CONFIGURATIONS
%----------------------------------------------------------------------------------------

\documentclass[10pt]{article} % Default font size
\usepackage{ragged2e}
\input{structure.tex} % Include the file specifying document layout

%----------------------------------------------------------------------------------------

\begin{document}

%----------------------------------------------------------------------------------------
%	NAME AND CONTACT INFORMATION
%----------------------------------------------------------------------------------------

\title{CV -- Alexandre Mercier-Aubin} % Print the main header

%------------------------------------------------

\parbox{0.5\textwidth}{ % Second block
\begin{tabbing} % Enables tabbing
\hspace{3cm} \= \hspace{4cm} \= \kill % Spacing within the block
{\bf Phone} \> +1 (418) 572 0698 \\ % Mobile phone
{\bf Email} \> \href{mailto:alexandre.mercier-aubin@etsmtl.ca}{alexandre.mercier-aubin@etsmtl.ca} \\ % Email address
{\bf Website} \> \href{https://alexandremercieraubin.com}{alexandremercieraubin.com} \\
\end{tabbing}}
\hfill % Horizontal space between the two blocks
\parbox{0.5\textwidth}{ % First block
\begin{tabbing} % Enables tabbing
\hspace{3cm} \= \hspace{4cm} \= \kill % Spacing within the block
{\bf Nationality} \> Canadian \\ % Nationality
{\bf Province} \> {Québec}\\
\end{tabbing}}

\vspace{-0.7cm} \centering{ {\bf Google Scholar} \> \href{https://scholar.google.ca/citations?user=N3Yv5IcAAAAJ}{https://scholar.google.ca/citations?user=N3Yv5IcAAAAJ}  }

\justifying

%----------------------------------------------------------------------------------------
%	PERSONAL PROFILE
%----------------------------------------------------------------------------------------

\section{Personal Profile}

My fields of study include computer graphics, physical simulation, optimization and algorithm design, as well as constraint programming. My field of research allows me to choose abstract topics while visualizing the results interactively. My work in computer graphics combines a certain artistic, sometimes even creative, level with the desire to advance the field. My results have led to applications both in surgical simulators and in more recreational contexts such as films and video games. I also have an interest for teaching, seeing this as an important step towards the  transfer of knowledge and skills to the new generation of workers and scientists.


%----------------------------------------------------------------------------------------
%	EDUCATION SECTION
%----------------------------------------------------------------------------------------

\section{Education}

\tabbedblock{
\bf{2020-2024} \> PhD in Computer Science - \href{https://www.ulaval.ca/en/}{McGill University} \\[3pt]
\>\+
\textit{Research in computer graphics on efficient animations with Prof P.G. Kry} \> \\
GPA: 3.90
}

%------------------------------------------------

\tabbedblock{
\bf{2019-2020} \> M. Sc. in Computer Science - \href{https://www.ulaval.ca/en/}{Université Laval} \\[3pt]
\>\+
\textit{Research in constraint programming with Prof. Claude-Guy Quimper} \> \\
GPA: 3.93
}

\tabbedblock{
\bf{2017-2019} \> B. Sc. in Computer Science - \href{https://www.ulaval.ca/en/}{Université Laval}\\[3pt]
\>\+
\textit{honors} \> \\
GPA: 3.73
}

\tabbedblock{
\bf{2014-2017} \> DEC in Computer Science - \href{http://cll.qc.ca/}{Cégep Lévis-Lauzon}
}

%----------------------------------------------------------------------------------------
%	EMPLOYMENT HISTORY SECTION
%----------------------------------------------------------------------------------------

\section{Teaching}
\job
{2020  -}{2023}
{McGill}
{mcgill.ca}
{teaching assistant and AGSEM delegate}
{Hold office hours, mark exams, and present various guest lectures for the courses on computer graphics, computer animations, and introduction to computer systems.}

\job
{}{2020}
{Université Laval}
{https://www.ulaval.ca/en/}
{teaching assistant}
{Teach the practical aspect (weekly labs) of the advanced programming in C++ course. Mark the exams in the computer graphics course. Support students at the help centre for computer science students (CARÉ) with questions related to 15 different computer science courses.}

\section{Research}
\job
{2025/01}{2025/06}
{École de technologie supérieure}
{https://www.etsmtl.ca/}
{Postdoctorate}
{Supervising graduate students, teaching, and grants writing.}


\job
{2024/05 -}{2024/07}
{Autodesk}
{Autodesk.com}
{Research Intern}
{Rigid body differentiable simulations for surface optimization.}

\job
{2017/01 -}{2017/08}
{Centre de Robotique et Vision Industrielle}
{http://www.crvi.ca/}
{Intern/Programmer}
{Machine learning applied to Computer Vision, program robot controllers, and develop a new website for employees.}

\section{Industry}

\job
{2018/05 -}{2018/09}
{Activision, Beenox}
{https://www.activision.com/}
{game engine developer intern}
{Design and program the game engine for Call of Duty: Black Ops 4.\\
 Technologies: DirectX, C++, LUA \\
 Contributions: Shaders, LOD formulas, Bugfixes, HUD formulas, cross-platform compatibility, etc.}

\job
{2016/05 -}{2016/09}
{Valero, Levis}
{https://www.valero.com}
{computer science intern}
{Translate programs from Visual Basic to C\#.}

\job
{2015/05 -}{2015/09}
{Consortium de ressources et d'expertises coopératives}
{https://leconsortium.coop/en/}
{IT Technician}
{IT support, creating and managing a database, creating a web site, etc.}

%----------------------------------------------------------------------------------------
%	IT/COMPUTING SKILLS SECTION
%----------------------------------------------------------------------------------------

\section{Prizes and Scholarships}

\skillgroup{Fonds de recherche du Québec (FRQNT): 2nd and 3rd cycle scholarship}
{
 25 000\$ \textit{per year, up to 3 years} \\
}

\skillgroup{Natural Sciences and Engineering Research Council of Canada (NSERC) Alliance Grant}
{
 15 000 Symgery partnership \\
}

\skillgroup{Bourse de doctorat Hydro-Québec en Science}
{
 15 000\$ \textit{per year, up to 2 years} \\
}

\skillgroup{School of Computer Science PhD funding, McGill University}
{
 21 000\$ \textit{per year, up to 3 years} \\
 3 000\$ top up from Mechanical Engineering.\\
}

\skillgroup{MITACS accelerate, CRISI}
{
\textit{} 39 000\$\\
}

\skillgroup{Undergraduate Research Fellowship 2019-2020, Université Laval}
{
\textit{I declined} 6500\$\\
}
%------------------------------------------------

\skillgroup{Association for Constraint Programming, CP2019}
{
\textit{} 450\$\\
}

\section{Publications}
In order to offer free and open access to scientific innovations, all my publications are listed free of charge on my website: \href{https://alexandremercieraubin.com/Work}{alexandremercieraubin.com/Work}

\newcounter{listCounter}
\subsection{Papers}
%\nocite{*}
%\bibliographystyle{acm}
%\bibliography{references}
\begin{enumerate}
  \setcounter{enumi}{\value{listCounter}}
  \item \textbf{Alexandre Mercier-Aubin}, Ludwig Dumetz, Jonathan Gaudreault, and Claude-Guy Quimper. The Confidence Constraint: A Step Towards Stochastic CP Solvers. In Proceedings of the 26th International Conference on Principles and Practice of Constraint Programming (CP), pages 759-773, 2020. \stepcounter{listCounter}

 \item \textbf{Alexandre Mercier-Aubin}, Jonathan Gaudreault, and Claude-Guy Quimper. Leveraging Constraint Scheduling: A Case Study to the Textile Industry. In Proceedings of the 17th International Conference on the Integration of Constraint Programming, Artificial Intelligence, and Operations Research (CPAIOR), pages 334-346, 2020. \stepcounter{listCounter}

 \item \textbf{Alexandre Mercier-Aubin},  Alexandre Winter,  David I. W. Levin, and Paul G. Kry. Adaptive Rigidification of Elastic Solids. In ACM Transactions on Graphics (TOG), volume 41, issue 4, article 71, 2022.  \stepcounter{listCounter}

 \item \textbf{Alexandre Mercier-Aubin} and Paul G. Kry. Adaptive Rigidification of Discrete Shells. In Proceedings of the ACM on Computer Graphics and Interactive Techniques (PACMCGIT), volume 6, issue 3, 2023. \stepcounter{listCounter}

\item \textbf{Alexandre Mercier-Aubin} and Paul G. Kry. A Multi-layer Solver for XPBD. In Proceedings of the Computer Graphics Forum (CGF), volume 43, issue 8, 2024. \stepcounter{listCounter}
\end{enumerate}

\subsection{Workshops}
\begin{enumerate}
  \setcounter{enumi}{\value{listCounter}}
  \item \textbf{Alexandre Mercier-Aubin}, Jonathan Gaudreault, and Claude-Guy Quimper. Multi-Resource Scheduling with Setup Times:
An Application Case to the Textile Industry. In Doctoral Program Proceedings of the 25th International
Conference on Principles and Practice of Constraint Programming (CP), 2019. \stepcounter{listCounter}

\end{enumerate}

\subsection{Master's Thesis}
\begin{enumerate}
  \setcounter{enumi}{\value{listCounter}}
  \item \textbf{Alexandre Mercier-Aubin}, Ordonnancement de tâches sous contraintes sur des métiers à tisser, Université Laval, 2020. \stepcounter{listCounter}
\end{enumerate}

\subsection{Posters}
\begin{enumerate}
  \setcounter{enumi}{\value{listCounter}}
  \item \textbf{Alexandre Mercier-Aubin}, Adaptive Rigidification of Elastic Solids Prototype, Graphics Interface (GI), 2022. \stepcounter{listCounter}
  \item \textbf{Alexandre Mercier-Aubin}, Adaptive Rigidification of Elastic Solids Prototype, colloque REPARTI, 2022. \stepcounter{listCounter}
\end{enumerate}

\subsection{Talks}
\begin{enumerate}
  \setcounter{enumi}{\value{listCounter}}
  \item The Confidence Constraint: A Step Towards Stochastic CP Solvers. International Conference on Principles and Practice of Constraint Programming (CP), 2020.\stepcounter{listCounter}
  \item  Leveraging Constraint Scheduling: A Case Study to the Textile Industry. International Conference on the Integration of Constraint Programming (CPAIOR), 2020.\stepcounter{listCounter}
  \item Adaptive Method for Soft Body Simulations. Tomatograph, 2021.\stepcounter{listCounter}
  \item Adaptive Rigidification of Elastic Solids. Special Interest Group on Computer Graphics and Interactive Techniques (SIGGRAPH), 2022. \stepcounter{listCounter}
  \item Infographie et Animation Physique : Solidification de Matériaux Viscoélatisques.  Séminaire Université Laval, 2022.\stepcounter{listCounter}
  \item Adaptive Rigidification of Discrete Shells. Symposium on Computer Animation (SCA), 2023. \stepcounter{listCounter}
  \item Strain-based Multi-Layer solver for XPBD. Quebec-Ontario pre-SIGGRAPH (GraphQuOn), 2023.\stepcounter{listCounter}
  \item A Multi-layer Solver for XPBD. Symposium on Computer Animation (SCA), 2024. \stepcounter{listCounter}
\end{enumerate}

%----------------------------------------------------------------------------------------
%	INTERESTS SECTION
%----------------------------------------------------------------------------------------

\section{Leadership}
\skillgroup{Undergraduate Projects}
{
\href{https://github.com/AlexandreMercierAubin/Gaudrophone}{Gaudrophone}, \href{https://github.com/AlexandreMercierAubin/BigData7027}{BigData7027}, and \href{https://github.com/AlexandreMercierAubin/ComputerGraphics}{ComputerGraphics}, etc. 
}
\skillgroup{Valleyfield Game Jam}
{}

\section{Other Projects}

\skillgroup{Engines}
{
-\href{https://github.com/AlexandreMercierAubin/AdaptiveRigidification2022}{Adaptive Rigidification Engine}\\
-\href{https://github.com/AlexandreMercierAubin/ComputerGraphics}{A simple computer graphics engine}\\

}

\skillgroup{Video Games}
{
-\href{https://www.callofduty.com/ca/en/blackops4}{Call of Duty: Black Ops 4}

-\href{https://youtu.be/qJjy8b0kuSY}{Proto-Spyder Assault, 48h Valleyfield game dev contest}

-\href{https://youtu.be/s6vr07Nt1IY}{SansFin, french horror game, Cegep school project}
}

\section{Service}
\textbf{AGSEM Delegate}: Delegate of the Computer Science department at McGill.

\noindent\textbf{V.P. Social at the AGIL}: Organize events for the association of graduate student in computer science.

\noindent\textbf{Volunteer at the ASETIN}: Volunteer work at the student association of computer science. 

\noindent\textbf{Volunteer at Lévis}: Receptionist at the Water Festival of Lévis.

\noindent\textbf{Student volunteer at SCA 2020 and 2024}: Support the conference by ensuring the smooth operation of sessions, main exhibitions, as well as the overall organization of activities and the design of promotional materials.

\noindent\textbf{Chair of session at SCA 2024}:  I was the chair of the Physics I: Fluids, Shells and Natural Phenomena session.

\noindent\textbf{Reviewer}: review papers for IEEE Transactions on Visualization and Computer Graphics (TVCG)and Eurographics (EG).

%----------------------------------------------------------------------------------------
%	REFEREE SECTION
%----------------------------------------------------------------------------------------

\section{Referees}

%\parbox{0.5\textwidth}{ 
%\begin{tabbing}
%\hspace{2.75cm} \= \hspace{4cm} \= \kill 
%{\bf Name} \> Claude-Guy Quimper \\ 
%{\bf Company} \> Université Laval \\ 
%{\bf Position} \> Associate Professor \\  
%{\bf Contact} \> \href{mailto:Claude-Guy.Quimper@ift.ulaval.ca}{Claude-Guy.Quimper@ift.ulaval.ca} 
%\end{tabbing}}\\

\noindent\parbox{0.5\textwidth}{
\begin{tabbing}
\hspace{2.75cm} \= \hspace{4cm} \= \kill
{\bf Name} \> Paul G. Kry\\ 
{\bf Company} \> McGill\\ 
{\bf Position} \> Associate Professor \\ 
{\bf Contact} \> \href{mailto:kry@cs.mcgill.ca}{kry@cs.mcgill.ca}
\end{tabbing}}

\noindent\parbox{0.5\textwidth}{
\begin{tabbing}
\hspace{2.75cm} \= \hspace{4cm} \= \kill
{\bf Name} \> Sheldon Andrews\\ 
{\bf Company} \> École de technologie supérieure\\ 
{\bf Position} \> Associate Professor \\ 
{\bf Contact} \> \href{mailto:sheldon.andrews@etsmtl.ca}{sheldon.andrews@etsmtl.ca}
\end{tabbing}}

\noindent\parbox{0.5\textwidth}{
\begin{tabbing}
\hspace{2.75cm} \= \hspace{4cm} \= \kill
{\bf Name} \> David I.W. Levin\\ 
{\bf Company} \> University of Toronto\\ 
{\bf Position} \>Associate Professor \\ 
{\bf Contact} \> \href{mailto:diwlevin@cs.toronto.edu}{diwlevin@cs.toronto.edu}
\end{tabbing}}


%----------------------------------------------------------------------------------------

\end{document}
