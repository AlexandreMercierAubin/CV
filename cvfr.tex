%----------------------------------------------------------------------------------------
%	PACKAGES AND OTHER DOCUMENT CONFIGURATIONS
%----------------------------------------------------------------------------------------

\documentclass[10pt]{article} % Default font size

\usepackage{ragged2e}
\usepackage{etoolbox}
\usepackage{fontspec}
\setmainfont{Times New Roman}

\patchcmd{\thebibliography}{\section*{\refname}}{}{}{}

%%%%%%%%%%%%%%%%%%%%%%%%%%%%%%%%%%%%%%%%%
% Wilson Resume/CV
% Structure Specification File
% Version 1.0 (22/1/2015)
%
% This file has been downloaded from:
% http://www.LaTeXTemplates.com
%
% License:
% CC BY-NC-SA 3.0 (http://creativecommons.org/licenses/by-nc-sa/3.0/)
%
%%%%%%%%%%%%%%%%%%%%%%%%%%%%%%%%%%%%%%%%%

%----------------------------------------------------------------------------------------
%	PACKAGES AND OTHER DOCUMENT CONFIGURATIONS
%----------------------------------------------------------------------------------------

\usepackage[a4paper, hmargin=25mm, vmargin=30mm, top=20mm]{geometry} % Use A4 paper and set margins

\usepackage{fancyhdr} % Customize the header and footer

\usepackage{lastpage} % Required for calculating the number of pages in the document

\usepackage{hyperref} % Colors for links, text and headings

\setcounter{secnumdepth}{0} % Suppress section numbering

%\usepackage[proportional,scaled=1.064]{erewhon} % Use the Erewhon font
%\usepackage[erewhon,vvarbb,bigdelims]{newtxmath} % Use the Erewhon font
\usepackage[utf8]{inputenc} % Required for inputting international characters
\usepackage[T1]{fontenc} % Output font encoding for international characters

\usepackage{fontspec} % Required for specification of custom fonts
\setmainfont[Path = ./fonts/,
Extension = .otf,
BoldFont = Erewhon-Bold,
ItalicFont = Erewhon-Italic,
BoldItalicFont = Erewhon-BoldItalic,
SmallCapsFeatures = {Letters = SmallCaps}
]{Erewhon-Regular}

\usepackage{color} % Required for custom colors
\definecolor{slateblue}{rgb}{0.17,0.22,0.34}

\usepackage{sectsty} % Allows customization of titles
\sectionfont{\color{slateblue}} % Color section titles

\fancypagestyle{plain}{\fancyhf{}\cfoot{\thepage\ of \pageref{LastPage}}} % Define a custom page style
\pagestyle{plain} % Use the custom page style through the document
\renewcommand{\headrulewidth}{0pt} % Disable the default header rule
\renewcommand{\footrulewidth}{0pt} % Disable the default footer rule

\setlength\parindent{0pt} % Stop paragraph indentation

% Non-indenting itemize
\newenvironment{itemize-noindent}
{\setlength{\leftmargini}{0em}\begin{itemize}}
{\end{itemize}}

% Text width for tabbing environments
\newlength{\smallertextwidth}
\setlength{\smallertextwidth}{\textwidth}
\addtolength{\smallertextwidth}{-2cm}

\newcommand{\sqbullet}{~\vrule height 1ex width .8ex depth -.2ex} % Custom square bullet point definition

%----------------------------------------------------------------------------------------
%	MAIN HEADER COMMAND
%----------------------------------------------------------------------------------------

\renewcommand{\title}[1]{
{\huge{\color{slateblue}\textbf{#1}}}\\ % Header section name and color
\rule{\textwidth}{0.5mm}\\ % Rule under the header
}

%----------------------------------------------------------------------------------------
%	JOB COMMAND
%----------------------------------------------------------------------------------------

\newcommand{\job}[6]{
\begin{tabbing}
\hspace{2cm} \= \kill
\textbf{#1} \> \href{#4}{#3} \\
\textbf{#2} \>\+ \textit{#5} \\
\begin{minipage}{\smallertextwidth}
\vspace{2mm}
#6
\end{minipage}
\end{tabbing}
\vspace{2mm}
}

%----------------------------------------------------------------------------------------
%	SKILL GROUP COMMAND
%----------------------------------------------------------------------------------------

\newcommand{\skillgroup}[2]{
\begin{tabbing}
\hspace{5mm} \= \kill
\sqbullet \>\+ \textbf{#1} \\
\begin{minipage}{\smallertextwidth}
\vspace{2mm}
#2
\end{minipage}
\end{tabbing}
}

%----------------------------------------------------------------------------------------
%	INTERESTS GROUP COMMAND
%-----------------------------------------------------------------------------------------

\newcommand{\interestsgroup}[1]{
\begin{tabbing}
\hspace{5mm} \= \kill
#1
\end{tabbing}
\vspace{-10mm}
}

\newcommand{\interest}[1]{\sqbullet \> \textbf{#1}\\[3pt]} % Define a custom command for individual interests

%----------------------------------------------------------------------------------------
%	TABBED BLOCK COMMAND
%----------------------------------------------------------------------------------------

\newcommand{\tabbedblock}[1]{
\begin{tabbing}
\hspace{2cm} \= \hspace{4cm} \= \kill
#1
\end{tabbing}
} % Include the file specifying document layout

%----------------------------------------------------------------------------------------

\begin{document}

%----------------------------------------------------------------------------------------
%	NAME AND CONTACT INFORMATION
%----------------------------------------------------------------------------------------

\title{CV -- Alexandre Mercier-Aubin} % Print the main header

%------------------------------------------------

\parbox{0.5\textwidth}{ % Second block
\begin{tabbing} % Enables tabbing
\hspace{3cm} \= \hspace{4cm} \= \kill % Spacing within the block
{\bf Téléphone} \> +1 (418) 572 0698 \\ % Mobile phone
{\bf Courriel} \> \href{mailto:alexandre.mercier-aubin@mail.mcgill.ca}{alexandre.mercier-aubin@mail.mcgill.ca} \\ % Email address
{\bf Site} \> \href{https://alexandremercieraubin.com}{alexandremercieraubin.com} \\
\end{tabbing}}
\hfill % Horizontal space between the two blocks
\parbox{0.5\textwidth}{ % First block
\begin{tabbing} % Enables tabbing
\hspace{3cm} \= \hspace{4cm} \= \kill % Spacing within the block
{\bf Nationalité} \> Canadien \\ % Nationality
{\bf Province} \> {Québec}\\
\end{tabbing}}

\vspace{-0.7cm} \centering{ {\bf Google Scholar} \> \href{https://scholar.google.ca/citations?user=N3Yv5IcAAAAJ}{https://scholar.google.ca/citations?user=N3Yv5IcAAAAJ}  }

\justifying
%----------------------------------------------------------------------------------------
%	PERSONAL PROFILE
%----------------------------------------------------------------------------------------

\section{Description}

Je suis candidat au doctorat avec l'intention de déposer ma thèse en \textbf{octobre 2024}. Mes domaines d'études incluent l'infographie, la simulation physique, l'optimisation et la conception d'algorithmes. Mon domaine de recherche me permet de choisir des sujets abstraits tout en visualisant les résultats de manière interactive. Mon travail en infographie combine un certain niveau artistique, parfois même créatif, avec la volonté de faire progresser le domaine. Mes résultats ont mené à des applications tant dans les simulateurs de chirurgie que dans des contextes plus ludiques comme les films et les jeux vidéo.


%----------------------------------------------------------------------------------------
%	EDUCATION SECTION
%----------------------------------------------------------------------------------------

\section{Éducation}

\tabbedblock{
\bf{2020 -} \> Doctorat en  Informatique - \href{https://www.ulaval.ca/en/}{McGill University} \\[3pt]
\bf{Présent} \>\+
\textit{Infographie et animations physiques avec Prof. Paul G. Kry} \\
GPA: 3.90
}

\tabbedblock{
\bf{2019 -} \> Maîtrise en  Informatique - \href{https://www.ulaval.ca/en/}{Université Laval} \\[3pt]
\bf{2020} \>\+
\textit{Programmation par contraintes avec Prof. Claude-Guy Quimper et Prof. Jonathan Gaudreault} \> \\
GPA: 3.93
}

\tabbedblock{
\bf{2017 -} \> Baccalauréat en Informatique - \href{https://www.ulaval.ca/en/}{Université Laval}\\[3pt]
\bf{2019} \>\+
\textit{honneurs} \> \\
GPA: 3.73
}

\tabbedblock{
\bf{2014 -} \> DEC en Informatique - \href{http://cll.qc.ca/}{Cégep Lévis-Lauzon}\\[3pt]
\bf{2017}
}

%----------------------------------------------------------------------------------------
%	EMPLOYMENT HISTORY SECTION
%----------------------------------------------------------------------------------------

\section{Enseignement}


\job
{2020 -}{2023}
{McGill}
{mcgill.ca}
{Auxiliaire d'enseignement et délégué AGSEM}
{Répondre aux questions des étudiants, corriger les examens et présenter les cours d'infographie,
d'animations et d'introduction aux systèmes informatiques lors de l'absence du professeur.}

\job
{2019 -}{2020}
{Université Laval}
{https://www.ulaval.ca/en/}
{Auxiliaire d'enseignement}
{Enseigner l'aspect pratique (laboratoires hebdomadaires) du cours de programmation avancée en C++. Corriger les examens du cours d'infographie. Soutenir les étudiants du centre d'appui à la réussite étudiante (CARÉ) avec des questions liées à 15 cours différents d'informatique.}

\section{Industrie}

\job
{2024/05}{2024/07}
{Autodesk}
{Autodesk.com}
{Stagaire en recherche}
{Création de simulation différentielle pour l'optimisation de paramètres et surfaces d'objets à transformation affine.}


\job
{2018/05,}{2018/09}
{Activision, Beenox}
{https://www.activision.com/}
{Développeur de moteurs graphiques}
{Concevoir et programmer le moteur de jeu pour Call of Duty : Black Ops 4.\\
 Technologies: DirectX, C++, LUA \\
 Contributions : Shaders, formules LOD, corrections de bogues, formules HUD, compatibilité multiplateforme, etc.}


\job
{2017/01 -}{2017/08}
{Centre de Robotique et Vision Industrielle}
{http://www.crvi.ca/}
{Programmeur}
{Apprentissage machine appliqué à la vision par ordinateur, programmation des contrôleurs de robots et développement d'un nouveau
site Web pour les employés.}

\job
{2016/05 -}{2016/09}
{Valero, Levis}
{https://www.valero.com}
{Stagiaire en informatique}
{Convertir des programmes de Visual Basic vers C\#.}

\job
{2015/05 -}{2015/09}
{Consortium de ressources et d'expertises coopératives}
{https://leconsortium.coop/en/}
{Technicien IT}
{Support informatique, création et gestion d'une base de données, création d'un site web, etc.}

%----------------------------------------------------------------------------------------
%	IT/COMPUTING SKILLS SECTION
%----------------------------------------------------------------------------------------

\section{Financement et Bourses}

\skillgroup{Fonds de recherche du Québec (FRQNT) 2e et 3e cycles}
{
 25 000\$  \textit{pendant 3 ans}\\
}

\skillgroup{Conseil de recherches en sciences naturelles et en génie du Canada (CRSNG) Alliance }
{
 15 000 - partenariat avec la compagnie Symgery Montréal\\
}

\skillgroup{Bourse de doctorat Hydro-Québec en Science}
{
 15 000\$ \textit{pendant 2 ans} \\
}

\skillgroup{Financement de doctorat McGill}
{
 21 000\$ \textit{pendant 3 ans} \\
 3 000\$ extra du département d'ingénierie mécanique\\
}

\skillgroup{MITACS accélération, CRISI, Université Laval}
{
\textit{} 39 000\$\\
}

%------------------------------------------------

\skillgroup{Association for Constraint Programming, CP2019}
{
\textit{} 450\$\\
}

\section{Recherche}
Dans le but d'offrir un accès ouvert et gratuit aux innovations scientifiques, toutes mes publications sont répertoriées gratuitement sur mon site internet : \href{https://alexandremercieraubin.com/Work}{alexandremercieraubin.com/Work}. Il est à noter que les articles d'infographie sont dans les journaux associés aux conférences Special Interest Group on Computer Graphics and Interactive Techniques (SIGGRAPH) ou de ACM SIGGRAPH / EUROGRAPHICS Symposium on Computer Animation (SCA).
\newcounter{listCounter}
\subsection{Articles}
%\nocite{*}
%\bibliographystyle{acm}
%\bibliography{references}
\begin{enumerate}
  \setcounter{enumi}{\value{listCounter}}
  \item \textbf{Alexandre Mercier-Aubin}, Ludwig Dumetz, Jonathan Gaudreault, and Claude-Guy Quimper. The Confidence Constraint: A Step Towards Stochastic CP Solvers. In Proceedings of the 26th International Conference on Principles and Practice of Constraint Programming (CP), pages 759-773, 2020. \stepcounter{listCounter}

 \item \textbf{Alexandre Mercier-Aubin}, Jonathan Gaudreault, and Claude-Guy Quimper. Leveraging Constraint Scheduling: A Case Study to the Textile Industry. In Proceedings of the 17th International Conference on the Integration of Constraint Programming, Artificial Intelligence, and Operations Research (CPAIOR), pages 334-346, 2020. \stepcounter{listCounter}

 \item \textbf{Alexandre Mercier-Aubin},  Alexandre Winter,  David I. W. Levin, and Paul G. Kry. Adaptive Rigidification of Elastic Solids. In ACM Transactions on Graphics (TOG), volume 41, issue 4, article 71, 2022.  \stepcounter{listCounter}

 \item \textbf{Alexandre Mercier-Aubin} and Paul G. Kry. Adaptive Rigidification of Discrete Shells. In Proceedings of the ACM on Computer Graphics and Interactive Techniques (PACMCGIT), volume 6, issue 3, 2023. \stepcounter{listCounter}

\item \textbf{Alexandre Mercier-Aubin} and Paul G. Kry. A Multi-layer Solver for XPBD. In Proceedings of the Computer Graphics Forum (CGF), volume 43, issue 8, 2024. \stepcounter{listCounter}
\end{enumerate}

\subsection{Ateliers}
\begin{enumerate}
  \setcounter{enumi}{\value{listCounter}}
  \item \textbf{Alexandre Mercier-Aubin}, Jonathan Gaudreault, and Claude-Guy Quimper. Multi-Resource Scheduling with Setup Times:
An Application Case to the Textile Industry. In  Doctoral Program Proceedings of the 25th International
Conference on Principles and Practice of Constraint Programming (CP), 2019. \stepcounter{listCounter}

\end{enumerate}

\subsection{Mémoire de maîtrise}
\begin{enumerate}
  \setcounter{enumi}{\value{listCounter}}
  \item \textbf{Alexandre Mercier-Aubin}, Ordonnancement de tâches sous contraintes sur des métiers à tisser, Université Laval, 2020. \stepcounter{listCounter}
\end{enumerate}

\subsection{Affiches}
\begin{enumerate}
  \setcounter{enumi}{\value{listCounter}}
  \item \textbf{Alexandre Mercier-Aubin}, Adaptive Rigidification of Elastic Solids Prototype, Graphics Interface (GI), 2022. \stepcounter{listCounter}
  \item \textbf{Alexandre Mercier-Aubin}, Adaptive Rigidification of Elastic Solids Prototype, colloque REPARTI, 2022. \stepcounter{listCounter}
\end{enumerate}

\subsection{Exposés}
\begin{enumerate}
  \setcounter{enumi}{\value{listCounter}}
  \item The Confidence Constraint: A Step Towards Stochastic CP Solvers. International Conference on Principles and Practice of Constraint Programming (CP), 2020.\stepcounter{listCounter}
  \item  Leveraging Constraint Scheduling: A Case Study to the Textile Industry. International Conference on the Integration of Constraint Programming (CPAIOR), 2020.\stepcounter{listCounter}
  \item Adaptive Method for Soft Body Simulations. Tomatograph, 2021.\stepcounter{listCounter}
  \item Adaptive Rigidification of Elastic Solids. Special Interest Group on Computer Graphics and Interactive Techniques (SIGGRAPH), 2022. \stepcounter{listCounter}
  \item Infographie et Animation Physique : Solidification de Matériaux Viscoélatisques.  Séminaire Université Laval, 2022.\stepcounter{listCounter}
  \item Adaptive Rigidification of Discrete Shells. Symposium on Computer Animation (SCA), 2023. \stepcounter{listCounter}
  \item Strain-based Multi-Layer solver for XPBD. Quebec-Ontario pre-SIGGRAPH (GraphQuOn), 2023.\stepcounter{listCounter}
  \item A Multi-layer Solver for XPBD. Symposium on Computer Animation (SCA), 2024. \stepcounter{listCounter}
\end{enumerate}

%----------------------------------------------------------------------------------------
%	INTERESTS SECTION
%----------------------------------------------------------------------------------------

\section{Autres Projets}

\skillgroup{Moteurs Graphiques}
{
-\href{https://github.com/AlexandreMercierAubin/AdaptiveRigidification2022}{Simulateur d'objets déformables pour rigidité adaptative}\\
-\href{https://github.com/AlexandreMercierAubin/ComputerGraphics}{Moteur graphique OpenGL/SDL2 simple.}\\

}

\skillgroup{Jeux Vidéo}
{
-\href{https://www.callofduty.com/ca/en/blackops4}{Call of Duty: Black Ops 4}

-\href{https://youtu.be/qJjy8b0kuSY}{Proto-Spyder Assault, Compétition de programmation de jeux vidéo en 48h à Valleyfield}

-\href{https://youtu.be/s6vr07Nt1IY}{SansFin, Projet de Cégep avec Unreal Engine}
}

\section{Service}
\textbf{Délégué AGSEM}: Délégué AGSEM du département d'informatique à McGill.

\noindent\textbf{V.P. des activités sociales AGIL}: Organiser des événements pour l'association des étudiants gradués en informatique de l'Université Laval.

\noindent\textbf{Volontaire à l 'ASETIN}: Bénévolat au sein de l'association étudiante d'informatique de l'Université Laval.

\noindent\textbf{Volontaire à la ville de Lévis}: Réceptionniste au Festival de l'Eau de Lévis.

\noindent\textbf{Volontaire étudiant SCA 2020 et 2024}: Soutenir la conférence en assurant le bon fonctionnement des opérations lors des
sessions de communications techniques et des principales expositions.

\noindent\textbf{Chair de séance SCA 2024}:   J'ai présidé la séance intitulée « Physics I: Fluids, Shells and Natural Phenomena.»

\noindent\textbf{Critique}: Critique d'articles scientifiques pour le journal IEEE Transactions on Visualization and Computer Graphics (TVCG).
%----------------------------------------------------------------------------------------
%	REFEREE SECTION
%----------------------------------------------------------------------------------------

\section{Références}

\parbox{0.5\textwidth}{
\begin{tabbing}
\hspace{2.75cm} \= \hspace{4cm} \= \kill
{\bf Nom} \> Paul G. Kry\\ 
{\bf Établissement} \> McGill University\\ 
{\bf Poste} \> Associate Professor \\ 
{\bf Courriel} \> \href{mailto:kry@cs.mcgill.ca}{kry@cs.mcgill.ca}
\end{tabbing}}

\noindent\parbox{0.5\textwidth}{
\begin{tabbing}
\hspace{2.75cm} \= \hspace{4cm} \= \kill
{\bf Nom} \> Sheldon Andrews\\ 
{\bf Établissement} \> École de technologie supérieure\\ 
{\bf Poste} \>  Professeur Agrégé \\ 
{\bf Courriel} \> \href{mailto:sheldon.andrews@etsmtl.ca}{sheldon.andrews@etsmtl.ca}
\end{tabbing}}

\noindent\parbox{0.5\textwidth}{
\begin{tabbing}
\hspace{2.75cm} \= \hspace{4cm} \= \kill
{\bf Nom} \> David I.W. Levin\\ 
{\bf Établissement} \> University of Toronto\\ 
{\bf Poste} \>Associate Professor \\ 
{\bf Courriel} \> \href{mailto:diwlevin@cs.toronto.edu}{diwlevin@cs.toronto.edu}
\end{tabbing}}


%----------------------------------------------------------------------------------------

\end{document}
