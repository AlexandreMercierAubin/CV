%----------------------------------------------------------------------------------------
%	PACKAGES AND OTHER DOCUMENT CONFIGURATIONS
%----------------------------------------------------------------------------------------

\documentclass[10pt]{article} % Default font size

\usepackage{ragged2e}

\input{structure.tex} % Include the file specifying document layout

%----------------------------------------------------------------------------------------

\begin{document}

%----------------------------------------------------------------------------------------
%	NAME AND CONTACT INFORMATION
%----------------------------------------------------------------------------------------

\title{CV -- Alexandre Mercier-Aubin} % Print the main header

%------------------------------------------------

\parbox{0.5\textwidth}{ % Second block
\begin{tabbing} % Enables tabbing
\hspace{3cm} \= \hspace{4cm} \= \kill % Spacing within the block
{\bf Téléphone} \> +1 (418) 572 0698 \\ % Mobile phone
{\bf Courriel} \> \href{mailto:alexandre.mercier-aubin@mail.mcgill.ca}{alexandre.mercier-aubin@mail.mcgill.ca} \\ % Email address
{\bf Site} \> \href{https://alexandremercieraubin.com}{alexandremercieraubin.com} \\
\end{tabbing}}
\hfill % Horizontal space between the two blocks
\parbox{0.5\textwidth}{ % First block
\begin{tabbing} % Enables tabbing
\hspace{3cm} \= \hspace{4cm} \= \kill % Spacing within the block
{\bf Nationalité} \> Canadien \\ % Nationality
{\bf Province} \> {Québec}\\
\end{tabbing}}

\vspace{-0.7cm} \centering{ {\bf Google Scholar} \> \href{https://scholar.google.ca/citations?user=N3Yv5IcAAAAJ}{https://scholar.google.ca/citations?user=N3Yv5IcAAAAJ}  }

\justifying
%----------------------------------------------------------------------------------------
%	PERSONAL PROFILE
%----------------------------------------------------------------------------------------

\section{Description}

Je suis candidat au doctorat avec l'intention de déposer ma thèse en automne 2024. Mes domaines d'études incluent l'infographie, la simulation physique, l'optimisation et la conception d'algorithmes, ainsi que la programmation par contraintes. Mon domaine de recherche me permet de choisir des sujets abstraits tout en visualisant les résultats de manière interactive. Mon travail en infographie combine un certain niveau artistique, parfois même créatif, avec la volonté de faire progresser le domaine. Mes résultats ont mené à des applications tant dans les simulateurs de chirurgie que dans des contextes plus ludiques comme les films et les jeux vidéo.

%----------------------------------------------------------------------------------------
%	EDUCATION SECTION
%----------------------------------------------------------------------------------------

\section{Éducation}

\tabbedblock{
\bf{2020-2024} \> Doctorat en  Informatique - \href{https://www.ulaval.ca/en/}{McGill University} \\[3pt]
\>\+
\textit{Infographie et animations physiques avec Prof. Paul G. Kry} \\
GPA: 3.90
}

%------------------------------------------------

\tabbedblock{
\bf{2019-2020} \> Maîtrise en  Informatique - \href{https://www.ulaval.ca/en/}{Université Laval} \\[3pt]
\>\+
\textit{Programmation par contraintes avec Prof. Claude-Guy Quimper} \> \\
GPA: 3.93
}

\tabbedblock{
\bf{2017-2019} \> Baccalauréat en Informatique - \href{https://www.ulaval.ca/en/}{Université Laval}\\[3pt]
\>\+
\textit{honneurs} \> \\
GPA: 3.73
}

\tabbedblock{
\bf{2014-2017} \> DEC en Informatique - \href{http://cll.qc.ca/}{Cégep Lévis-Lauzon}
}

%----------------------------------------------------------------------------------------
%	EMPLOYMENT HISTORY SECTION
%----------------------------------------------------------------------------------------

\section{Enseignement}


\job
{2020 -}{2023}
{McGill}
{mcgill.ca}
{Auxiliaire d'enseignement et délégué AGSEM}
{Répondre aux questions des étudiants, corriger les examens et présenter les cours d'infographie,
d'animations et d'introduction aux systèmes informatiques lors de l'absence du professeur.}

\job
{2019 -}{2020}
{Université Laval}
{https://www.ulaval.ca/en/}
{Auxiliaire d'enseignement}
{Enseigner l'aspect pratique (laboratoires hebdomadaires) du cours de programmation avancée en C++. Corriger les examens du cours d'infographie. Soutenir les étudiants du centre d'appui à la réussite étudiante (CARÉ) avec des questions liées à 15 cours différents d'informatique.}

\section{Industrie}

\job
{2024/05}{2024/07}
{Autodesk}
{Autodesk.com}
{Stagaire en recherche}
{Création de simulation différentielle pour l'optimisation de paramètres et surfaces.}


\job
{2018/05,}{2018/09}
{Activision, Beenox}
{https://www.activision.com/}
{Développeur de moteurs graphiques}
{Concevoir et programmer le moteur de jeu pour Call of Duty : Black Ops 4.\\
 Technologies: DirectX, C++, LUA \\
 Contributions : Shaders, formules LOD, corrections de bogues, formules HUD, compatibilité multiplateforme, etc.}


\job
{2017/01 -}{2017/08}
{Centre de Robotique et Vision Industrielle}
{http://www.crvi.ca/}
{Stagiaire/Programmeur}
{Apprentissage automatique appliqué à la vision par ordinateur, programmation des contrôleurs de robots et développement d'un nouveau
site Web pour les employés.}

\job
{2016/05 -}{2016/09}
{Valero, Levis}
{https://www.valero.com}
{Stagiaire en informatique}
{Convertir des programmes de Visual Basic vers C\#.}

\job
{2015/05 -}{2015/09}
{Consortium de ressources et d'expertises coopératives}
{https://leconsortium.coop/en/}
{Technicien IT}
{Support informatique, création et gestion d'une base de données, création d'un site web, etc.}

%----------------------------------------------------------------------------------------
%	IT/COMPUTING SKILLS SECTION
%----------------------------------------------------------------------------------------

\section{Financement et Bourses}

\skillgroup{Fonds de recherche du Québec (FRQNT) 2e et 3e cycles}
{
 25 000\$  \textit{pendant 3 ans}\\
}

\skillgroup{Conseil de recherches en sciences naturelles et en génie du Canada (CRSNG) Alliance }
{
 15 000 - partenariat avec la compagnie Symgery Montréal\\
}

\skillgroup{Bourse de doctorat Hydro-Québec en Science}
{
 15 000\$ \textit{pendant 2 ans} \\
}

\skillgroup{Financement de doctorat McGill}
{
 21 000\$ \textit{pendant 3 ans} \\
 3 000\$ extra du département d'ingénierie mécanique\\
}

\skillgroup{MITACS accélération, CRISI, Université Laval}
{
\textit{} 39 000\$\\
}

%------------------------------------------------

\skillgroup{Association for Constraint Programming, CP2019}
{
\textit{} 450\$\\
}

\section{Recherche}
Dans le but d'offrir un accès ouvert et gratuit aux innovations scientifiques, toutes mes publications sont répertoriées gratuitement sur mon site internet : \href{https://alexandremercieraubin.com/Work}{alexandremercieraubin.com/Work}

%----------------------------------------------------------------------------------------
%	INTERESTS SECTION
%----------------------------------------------------------------------------------------

\section{Autres Projets}

\skillgroup{Moteurs Graphiques}
{
-\href{https://github.com/AlexandreMercierAubin/AdaptiveRigidification2022}{Simulateur d'objets déformables pour rigidité adaptative}\\
-\href{https://github.com/AlexandreMercierAubin/ComputerGraphics}{Moteur graphique OpenGL/SDL2 simple.}\\

}

\skillgroup{Jeux Vidéos}
{
-\href{https://www.callofduty.com/ca/en/blackops4}{Call of Duty: Black Ops 4}

-\href{https://youtu.be/qJjy8b0kuSY}{Proto-Spyder Assault, Compétition de programmation de jeux vidéos en 48h à Valleyfield}

-\href{https://youtu.be/s6vr07Nt1IY}{SansFin, Projet de Cégep avec Unreal Engine}
}

\section{Service}
\textbf{Délégué AGSEM}: Délégué AGSEM du département d'informatique à McGill.

\noindent\textbf{V.P. des activités sociales AGIL}: Organiser des événements pour l'association des étudiants gradués en informatique de l'Université Laval.

\noindent\textbf{Volontaire à l 'ASETIN}: Bénévolat au sein de l'association étudiante d'informatique de l'Université Laval.

\noindent\textbf{Volontaire à la ville de Lévis}: Réceptionniste au Festival de l'Eau de Lévis.

\noindent\textbf{Volontaire étudiant SCA 2020 et 2024}: Soutenir la conférence en assurant le bon fonctionnement des opérations lors des
sessions de communications techniques et des principales expositions.

\noindent\textbf{Critique}: Critique d'articles scientifiques pour le journal IEEE Transactions on Visualization and Computer Graphics (TVCG).
%----------------------------------------------------------------------------------------
%	REFEREE SECTION
%----------------------------------------------------------------------------------------

\section{Références}

\parbox{0.5\textwidth}{
\begin{tabbing}
\hspace{2.75cm} \= \hspace{4cm} \= \kill
{\bf Nom} \> Paul G. Kry\\ 
{\bf Établissement} \> McGill University\\ 
{\bf Poste} \> Associate Professor \\ 
{\bf Courriel} \> \href{mailto:kry@cs.mcgill.ca}{kry@cs.mcgill.ca}
\end{tabbing}}

\noindent\parbox{0.5\textwidth}{
\begin{tabbing}
\hspace{2.75cm} \= \hspace{4cm} \= \kill
{\bf Nom} \> Sheldon Andrews\\ 
{\bf Établissement} \> École de technologie supérieure\\ 
{\bf Poste} \>  Professeur Agrégé \\ 
{\bf Courriel} \> \href{mailto:sheldon.andrews@etsmtl.ca}{sheldon.andrews@etsmtl.ca}
\end{tabbing}}

\noindent\parbox{0.5\textwidth}{
\begin{tabbing}
\hspace{2.75cm} \= \hspace{4cm} \= \kill
{\bf Nom} \> David I.W. Levin\\ 
{\bf Établissement} \> University of Toronto\\ 
{\bf Poste} \>Associate Professor \\ 
{\bf Courriel} \> \href{mailto:diwlevin@cs.toronto.edu}{diwlevin@cs.toronto.edu}
\end{tabbing}}


%----------------------------------------------------------------------------------------

\end{document}
