%----------------------------------------------------------------------------------------
%	PACKAGES AND OTHER DOCUMENT CONFIGURATIONS
%----------------------------------------------------------------------------------------

\documentclass[10pt]{article} % Default font size

\usepackage{ragged2e}

%%%%%%%%%%%%%%%%%%%%%%%%%%%%%%%%%%%%%%%%%
% Wilson Resume/CV
% Structure Specification File
% Version 1.0 (22/1/2015)
%
% This file has been downloaded from:
% http://www.LaTeXTemplates.com
%
% License:
% CC BY-NC-SA 3.0 (http://creativecommons.org/licenses/by-nc-sa/3.0/)
%
%%%%%%%%%%%%%%%%%%%%%%%%%%%%%%%%%%%%%%%%%

%----------------------------------------------------------------------------------------
%	PACKAGES AND OTHER DOCUMENT CONFIGURATIONS
%----------------------------------------------------------------------------------------

\usepackage[a4paper, hmargin=25mm, vmargin=30mm, top=20mm]{geometry} % Use A4 paper and set margins

\usepackage{fancyhdr} % Customize the header and footer

\usepackage{lastpage} % Required for calculating the number of pages in the document

\usepackage{hyperref} % Colors for links, text and headings

\setcounter{secnumdepth}{0} % Suppress section numbering

%\usepackage[proportional,scaled=1.064]{erewhon} % Use the Erewhon font
%\usepackage[erewhon,vvarbb,bigdelims]{newtxmath} % Use the Erewhon font
\usepackage[utf8]{inputenc} % Required for inputting international characters
\usepackage[T1]{fontenc} % Output font encoding for international characters

\usepackage{fontspec} % Required for specification of custom fonts
\setmainfont[Path = ./fonts/,
Extension = .otf,
BoldFont = Erewhon-Bold,
ItalicFont = Erewhon-Italic,
BoldItalicFont = Erewhon-BoldItalic,
SmallCapsFeatures = {Letters = SmallCaps}
]{Erewhon-Regular}

\usepackage{color} % Required for custom colors
\definecolor{slateblue}{rgb}{0.17,0.22,0.34}

\usepackage{sectsty} % Allows customization of titles
\sectionfont{\color{slateblue}} % Color section titles

\fancypagestyle{plain}{\fancyhf{}\cfoot{\thepage\ of \pageref{LastPage}}} % Define a custom page style
\pagestyle{plain} % Use the custom page style through the document
\renewcommand{\headrulewidth}{0pt} % Disable the default header rule
\renewcommand{\footrulewidth}{0pt} % Disable the default footer rule

\setlength\parindent{0pt} % Stop paragraph indentation

% Non-indenting itemize
\newenvironment{itemize-noindent}
{\setlength{\leftmargini}{0em}\begin{itemize}}
{\end{itemize}}

% Text width for tabbing environments
\newlength{\smallertextwidth}
\setlength{\smallertextwidth}{\textwidth}
\addtolength{\smallertextwidth}{-2cm}

\newcommand{\sqbullet}{~\vrule height 1ex width .8ex depth -.2ex} % Custom square bullet point definition

%----------------------------------------------------------------------------------------
%	MAIN HEADER COMMAND
%----------------------------------------------------------------------------------------

\renewcommand{\title}[1]{
{\huge{\color{slateblue}\textbf{#1}}}\\ % Header section name and color
\rule{\textwidth}{0.5mm}\\ % Rule under the header
}

%----------------------------------------------------------------------------------------
%	JOB COMMAND
%----------------------------------------------------------------------------------------

\newcommand{\job}[6]{
\begin{tabbing}
\hspace{2cm} \= \kill
\textbf{#1} \> \href{#4}{#3} \\
\textbf{#2} \>\+ \textit{#5} \\
\begin{minipage}{\smallertextwidth}
\vspace{2mm}
#6
\end{minipage}
\end{tabbing}
\vspace{2mm}
}

%----------------------------------------------------------------------------------------
%	SKILL GROUP COMMAND
%----------------------------------------------------------------------------------------

\newcommand{\skillgroup}[2]{
\begin{tabbing}
\hspace{5mm} \= \kill
\sqbullet \>\+ \textbf{#1} \\
\begin{minipage}{\smallertextwidth}
\vspace{2mm}
#2
\end{minipage}
\end{tabbing}
}

%----------------------------------------------------------------------------------------
%	INTERESTS GROUP COMMAND
%-----------------------------------------------------------------------------------------

\newcommand{\interestsgroup}[1]{
\begin{tabbing}
\hspace{5mm} \= \kill
#1
\end{tabbing}
\vspace{-10mm}
}

\newcommand{\interest}[1]{\sqbullet \> \textbf{#1}\\[3pt]} % Define a custom command for individual interests

%----------------------------------------------------------------------------------------
%	TABBED BLOCK COMMAND
%----------------------------------------------------------------------------------------

\newcommand{\tabbedblock}[1]{
\begin{tabbing}
\hspace{2cm} \= \hspace{4cm} \= \kill
#1
\end{tabbing}
} % Include the file specifying document layout

%----------------------------------------------------------------------------------------

\begin{document}

%----------------------------------------------------------------------------------------
%	NAME AND CONTACT INFORMATION
%----------------------------------------------------------------------------------------

\title{CV -- Alexandre Mercier-Aubin} % Print the main header

%------------------------------------------------

\parbox{0.5\textwidth}{ % Second block
\begin{tabbing} % Enables tabbing
\hspace{3cm} \= \hspace{4cm} \= \kill % Spacing within the block
{\bf Téléphone} \> +1 (418) 572 0698 \\ % Mobile phone
{\bf Email} \> \href{mailto:alexandre.mercier-aubin@mail.mcgill.ca}{alexandre.mercier-aubin@mail.mcgill.ca} \\ % Email address
{\bf Site} \> \href{https://alexandremercieraubin.com}{alexandremercieraubin.com} \\
\end{tabbing}}
\hfill % Horizontal space between the two blocks
\parbox{0.5\textwidth}{ % First block
\begin{tabbing} % Enables tabbing
\hspace{3cm} \= \hspace{4cm} \= \kill % Spacing within the block
{\bf Nationalité} \> Canadien \\ % Nationality
{\bf Province} \> {Québec}\\
\end{tabbing}}

\vspace{-0.7cm} \centering{ {\bf Google Scholar} \> \href{https://scholar.google.ca/citations?user=N3Yv5IcAAAAJ}{https://scholar.google.ca/citations?user=N3Yv5IcAAAAJ}  }

\justifying
%----------------------------------------------------------------------------------------
%	PERSONAL PROFILE
%----------------------------------------------------------------------------------------

\section{Description}

Je suis un candidat au doctorat tout sauf dissertation. Mes domaines d'études sont l'infographie, la simulation de physique, l'optimisation et la conception
d'algorithmes et la programmation par contraintes. Mon domaine de recherche me permet de choisir des sujets de recherche abstrait tout en visualizant les résultats de façon interactive. 
Mon travail sur en infographie combine un certain niveau artistique, même parfois créatif avec la volonté de faire avancer le domaine. Mes résultats ont menés à des applications autant au niveau de simulateurs de chirugie que dans des contextes plus ludiques comme les films et jeux vidéos.

%----------------------------------------------------------------------------------------
%	EDUCATION SECTION
%----------------------------------------------------------------------------------------

\section{Education}

\tabbedblock{
\bf{2020-2024} \> Doctorat en  Informatique - \href{https://www.ulaval.ca/en/}{McGill University} \\[3pt]
\>\+
\textit{Infographie et animations physiques avec Prof. Paul G. Kry} \\
GPA: 3.90
}

%------------------------------------------------

\tabbedblock{
\bf{2019-2020} \> Maîtrise en  Informatique - \href{https://www.ulaval.ca/en/}{Université Laval} \\[3pt]
\>\+
\textit{Programmation par contraintes avec Prof. Claude-Guy Quimper} \> \\
GPA: 3.93
}

\tabbedblock{
\bf{2017-2019} \> Baccalauréat en Informatique - \href{https://www.ulaval.ca/en/}{Université Laval}\\[3pt]
\>\+
\textit{honeurs} \> \\
GPA: 3.73
}

\tabbedblock{
\bf{2014-2017} \> DEC en Informatique - \href{http://cll.qc.ca/}{Cégep Lévis-Lauzon}
}

%----------------------------------------------------------------------------------------
%	EMPLOYMENT HISTORY SECTION
%----------------------------------------------------------------------------------------

\section{Enseignement}


\job
{}{2020-2023}
{McGill}
{mcgill.ca}
{Auxiliaire d'enseignement and délégué AGSEM}
{Répondre aux questions des étudiants, corriger les examens et présenter les cours d'infographie,
d'animations et d'introduction aux systèmes informatiques lors de l'absence du professeur.}

\job
{}{2019-2020}
{Université Laval}
{https://www.ulaval.ca/en/}
{Auxiliaire d'enseignement}
{Enseigner l'aspect pratique (laboratoires hebdomadaires) du cours de programmation avancée en C++. Corriger les examens du cours d'infographie. Soutenir les étudiants du Centre d'appui à la réussite étudiante (CARÉ) avec des questions liées à 15 cours différents d'informatique.}

\section{Industrie}

\job
{}{Été 2024}
{Autodesk}
{Autodesk.com}
{Stagaire en recherche}
{Création de simulation différentielle pour l'optimisation de paramètres et surfaces.}


\job
{2018/05,}{2018/09}
{Activision, Beenox}
{https://www.activision.com/}
{Développeur de moteurs graphiques}
{Concevoir et programmer le moteur de jeu pour Call of Duty : Black Ops 4.\\
 Technologies: DirectX, C++, LUA \\
 Contributions : Shaders, formules LOD, corrections de bugs, formules HUD, compatibilité multiplateforme, etc.}


\job
{2017/01 -}{2017/08}
{Centre de Robotic et Vision Industrielle}
{http://www.crvi.ca/}
{Intern/Programmer}
{Apprentissage automatique appliqué à la vision par ordinateur, programmation des contrôleurs de robots et développement d'un nouveau
site Web pour les employés.}

\job
{2016/05 -}{2016/09}
{Valero, Levis}
{https://www.valero.com}
{computer science intern}
{Convertir des programmes de Visual Basic vers C\#.}

\job
{2015/05 -}{2015/09}
{Consortium de ressources et d'expertises coopératives}
{https://leconsortium.coop/en/}
{IT Technician}
{Support informatique, création et gestion d'une base de données, création d'un site web, etc.}

%----------------------------------------------------------------------------------------
%	IT/COMPUTING SKILLS SECTION
%----------------------------------------------------------------------------------------

\section{Financement et Bourses}

\skillgroup{Fonds de recherche du Québec (FRQNT): 2nd and 3rd cycle scholarship}
{
 25 000\$  \textit{pendant 3 ans}\\
}

\skillgroup{Conseil de recherches en sciences naturelles et en génie du Canada (CRSNG) Alliance }
{
 15 000 - partenariat avec la compagnie Symgery Montréal\\
}

\skillgroup{Bourse de doctorat Hydro-Québec en Science}
{
 15 000\$ \textit{pendant 2 ans} \\
}

\skillgroup{Financement de doctorat McGill}
{
 21 000\$ \textit{pendant 3 ans} \\
 3 000\$ extra du département d'ingénierie mécanique\\
}

\skillgroup{MITACS accélération, CRISI}
{
\textit{} 39 000\$\\
}

%------------------------------------------------

\skillgroup{Association for Constraint Programming, CP2019}
{
\textit{} 450\$\\
}

\section{Recherche}
Toutes mes publications sont répertoriées gratuitement sur mon site internet : \href{https://alexandremercieraubin.com/Work}{alexandremercieraubin.com/Work}

%----------------------------------------------------------------------------------------
%	INTERESTS SECTION
%----------------------------------------------------------------------------------------

\section{Autre Projets}

\skillgroup{Moteurs Graphiques}
{
-\href{https://github.com/AlexandreMercierAubin/AdaptiveRigidification2022}{Simulateur d'objets déformables pour rigidité adaptive}\\
-\href{https://github.com/AlexandreMercierAubin/ComputerGraphics}{Moteur graphique OpenGL/SDL2 simple.}\\

}

\skillgroup{Jeux Vidéos}
{
-\href{https://www.callofduty.com/ca/en/blackops4}{Call of Duty: Black Ops 4}

-\href{https://youtu.be/qJjy8b0kuSY}{Proto-Spyder Assault, Compétition de programmation de jeux vidéos en 48h à Valleyfield}

-\href{https://youtu.be/s6vr07Nt1IY}{SansFin, Projet de Cégep avec Unreal Engine}
}

\section{Service}
\textbf{Délégué AGSEM}:Délégué AGSEM du département d'informatique à McGill

\noindent\textbf{V.P. des activitées sociales AGIL}: Organiser des événements pour l'association des étudiants diplômés en informatique.

\noindent\textbf{Volontaire à l 'ASETIN}: Bénévolat au sein de l'association étudiante d'informatique. J'ai été barman, réceptionniste et j'ai
aidé à préparer les initiations des étudiants

\noindent\textbf{Volontaire à la ville de Lévis}: Réceptionniste au Festival de l'Eau de Lévis.

\noindent\textbf{Volontaire étudiant SCA 2020 et 2024}: Soutenir la conférence en assurant le bon fonctionnement des opérations lors des
sessions de communications techniques et des principales expositions.
%----------------------------------------------------------------------------------------
%	REFEREE SECTION
%----------------------------------------------------------------------------------------

\section{Références}

\parbox{0.5\textwidth}{
\begin{tabbing}
\hspace{2.75cm} \= \hspace{4cm} \= \kill
{\bf Nom} \> Paul G. Kry\\ 
{\bf Établissement} \> McGill\\ 
{\bf Poste} \> Associate Professor \\ 
{\bf Email} \> \href{mailto:kry@cs.mcgill.ca}{kry@cs.mcgill.ca}
\end{tabbing}}

\noindent\parbox{0.5\textwidth}{
\begin{tabbing}
\hspace{2.75cm} \= \hspace{4cm} \= \kill
{\bf Nom} \> Sheldon Andrews\\ 
{\bf Établissement} \> École de technologie supérieure\\ 
{\bf Poste} \>  Professeur Agrégé \\ 
{\bf Email} \> \href{mailto:sheldon.andrews@etsmtl.ca}{sheldon.andrews@etsmtl.ca}
\end{tabbing}}

\noindent\parbox{0.5\textwidth}{
\begin{tabbing}
\hspace{2.75cm} \= \hspace{4cm} \= \kill
{\bf Nom} \> David I.W. Levin\\ 
{\bf Établissement} \> University of Toronto\\ 
{\bf Poste} \>Assistant Professor \\ 
{\bf Email} \> \href{mailto:diwlevin@cs.toronto.edu}{diwlevin@cs.toronto.edu}
\end{tabbing}}


%----------------------------------------------------------------------------------------

\end{document}
